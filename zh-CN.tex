\documentclass[12pt]{article}
\usepackage[margin=4em]{geometry}
\usepackage{titlesec}
\usepackage{enumitem}
\usepackage{hyperref}
\usepackage{blindtext}
\usepackage{libertine}
\usepackage{xeCJK}
\setmonofont{inconsolata}


%\usepackage[T1]{fontenc}
%\usepackage{fontspec}
\usepackage{enumitem}
\setitemize{noitemsep,topsep=6pt,parsep=1pt,partopsep=0pt}

\usepackage{titlesec}
\titlespacing{\section}{0pt}{10pt}{1em}

%\titleformat{\section}{\vspace{1em}}{\thesection.}{0.5em}{\vspace{0pt}}[{\titlerule[0.1pt]}]
%\titleformat{\subsection}{\it}{\thesubsection.}{0.5em}{\vspace{-0.5em}}[]
\setlength\parindent{0pt}
\pagestyle{empty}

%\setmainfont{CMU Concrete}

\begin{document}
{\Huge\textsf{何文琦}} \quad \texttt{github.com/galmungral}

\section*{Experience}

\textbf{美国国家超级计算机应用中心 (NCSA)}, Urbana, IL, US\\
\textit{研究生科研助理 / 科研软件工程师} (RSE) \textit{,可视化分析组, 视觉部门} \hfill Aug. 2022 - Present
\begin{itemize}
\item \texttt{为美国国立卫生研究院资助的科研项目开发软件, 专注于生物医学及地理空间数据的可视化.}
\item \texttt{使用 OpenLayers 和 GeoJSON 制作了伊利诺伊州蜱虫数据的交互式地图可视化系统.}
\item \texttt{参与了癌症演化树(phylogeny)可视化分析工具 PhyloDiver 的维护与开发.}
\end{itemize}

\textbf{谷露软件}, Shanghai, China\\
\textit{前端工程师, 申请人追踪系统} (ATS) \hfill Nov. 2020 - Nov. 2021
\begin{itemize}
\item \texttt{在多个项目团队内开发了多个web应用、微信小程序、前端库和后端图片渲染服务.}
\item \texttt{参与了高度可定制的ATS系统底层的拖拽式图形用户界面生成器的维护与开发.}
\item \texttt{负责公司首个单点登录系统(SSO)的前端界面和多个项目前端登陆流程的整合.}
\end{itemize}

\textbf{Étude LLC}, Atlanta, GA, US\\
\textit{软件工程师, 创始团队} \hfill Aug. 2019 - Aug. 2020
\begin{itemize}
\item \texttt{开发了智能PDF阅读器的多个主要功能, 包括准确率>85\%的图书目次自动解析器.}
\end{itemize}

\section*{Education}
\textbf{University of Illinois Urbana-Champaign}, Urbana–Champaign, IL, US \hfill Aug. 2022 - Dec. 2023 \\
\textit{计算机科学硕士} (MCS)  \hfill GPA: 4.0/4.0 
\begin{itemize}
\item \texttt{编程语言设计/形式语义学(K), 编译器 (LLVM), 自然语言处理, 信息检索, 分布式算法}
\item \texttt{交互式计算机图形学 (WebGL), 科学可视化(标/矢/张量场和生物信息可视化), 计算摄影学}
\end{itemize}

\textbf{Georgia Institute of Technology}, Atlanta, GA, US \hfill Aug. 2015 - Dec. 2019\\
\textit{计算机科学理学学士} (BSCS, \textit{summa cum laude}), \textit{辅修物理学} \hfill 专业 GPA: 4.0/4.0, GPA: 3.97/4.0
\begin{itemize}
\item \texttt{经典微分几何, 偏/常微分方程, 数值分析, 计算机模拟 (离散事件模拟), 广义相对论}
\end{itemize}

\section*{Personal Projects}
\href{https://galmungral.github.io/mercator}{\textit{简易栅格地图瓦片渲染器}} \hfill (TypeScript) \\
\href{https://galmungral.github.io/particle-simulation}{\textit{简易粒子碰撞模拟/可视化}} 
\hfill (WebGL, Rust, WebAssembly) \\
\href{https://galmungral.github.io/terrain-generator}{\textit{简易3D地形生成器}} \hfill (WebGL, Rust, WebAssembly) \\
\href{https://galmungral.github.io/rasterizer?file=billboard.txt}{\textit{简易3D软件栅格化器}} \hfill (C++, Emscripten, WebAssembly) \\
\href{https://galmungral.github.io/text2svg}{\textit{简易OpenType/TrueType字体渲染器}} \hfill (TypeScript) \\
\href{https://galmungral.github.io/michelangelo}{\textit{基于Canvas的简易 UI 框架/布局系统}} \hfill (TypeScript) \\
\href{https://github.com/galmungral/react-teletype}{\textit{服务端交互式 React 渲染器}} \hfill (TypeScript) \\
\href{https://github.com/galmungral/replay}{\textit{简易前端框架与模块打包器}} \hfill (TypeScript) \\
\href{https://galmungral.github.io/hanbun-lang}{\textit{基于文言的深奥(esoteric)编程语言}} \hfill (PureScript) \\
\href{https://github.com/galmungral/json-crdt}{\textit{JSON式无冲突复制数据类型(CRDT)的实用性实现}} \hfill (TypeScript) \\
\href{https://github.com/galmungral/telescope}{\textit{用以规避互联网审查的分离式SOCKS代理}} \hfill (C, libuv) \\
\href{https://github.com/galmungral/plato}{\textit{基于查询似然模型的简易搜索引擎}} \hfill (Python) \\
\href{https://github.com/galmungral/turing-machine}{\textit{通用图灵机的模拟}} \hfill (Haskell, Common Lisp)
\end{document}
