\documentclass[11pt,twocolumn]{article}
\setlength\columnsep{0pt}

\usepackage[margin=4em]{geometry}

\setlength\parindent{0pt}
\pagestyle{empty}

\usepackage{libertine}
\setmonofont{inconsolata}


\usepackage{enumitem}
\setitemize{itemsep=0.2em,topsep=1em,parsep=1pt,partopsep=0pt,leftmargin=2em}

\usepackage{titlesec}
\usepackage{fontspec}
\newfontfamily\cjkitalic{STFangsong}
\newfontfamily\cjkbold{STSongti-SC-Black}
\newfontfamily\cjkmain{STKaitiSC-Regular}


\titlespacing{\section}{-1em}{1em}{0.5em}
\titlespacing{\subsection}{-0.5em}{0.5em}{0.2em}

\usepackage{hyperref}
\usepackage{xeCJK}
% \setCJKmainfont{SongTi TC}
\setCJKmainfont[BoldFont=STSongti-SC-Black, ItalicFont=STFangsong]{STKaitiSC-Regular}


\newcommand{\comment}[1]{}


\begin{document}

{\Huge\bf 何文琦}

\vspace{10pt}
\href{mailto:hewenqi96@gmail.com}{\texttt{hewenqi96@gmail.com}}\\
\href{https://linkedin.com/in/galmungral}{\texttt{linkedin.com/in/galmungral}}\\
\href{https://github.com/galmungral}{\texttt{github.com/galmungral}}


\section*{\textnormal{EDUCATION}}

\subsection*{University of Illinois Urbana-Champaign}
\textit{计算机科学硕士}\\
Dec. 2023 \textbullet\ Urbana, IL, US

\subsection*{Georgia Institute of Technology}
\textit{计算机科学理学学士 (辅修物理学)}\\
Dec. 2019 \textbullet\ Atlanta, GA, US

\section*{\textnormal{PROJECTS}}
\href{https://github.com/galmungral/polyrender}{A GPU-accelerated 2D vector graphics renderer}\\
\href{https://github.com/galmungral/hanbun-lang}{A stack-oriented esoteric programming language} \\
\href{https://github.com/galmungral/mercator}{A raster map tile renderer}\\
\href{https://github.com/galmungral/michelangelo}{A canvas-based UI framework}\\
\href{https://github.com/galmungral/react-teletype}{A server-side interactive React renderer} \\
\href{https://github.com/galmungral/replay}{A frontend framework with a module bundler}

\section*{\textnormal{COURSEWORK}}

\subsection*{{Graphics / Physics}}
Scientific Visualization\\
Interactive Computer Graphics\\
Computational Photography\\
Parallel Programming\\
Numerical Analysis/Methods\\
Partial Differential Equations\\
Differential Geometry\\
General Relativity\\
Quantum Mechanics\\
Quantum Computing\\

\subsection*{{Languages / Communication}}
Compiler Construction\\
Programming Languages\\
Natural Language Processing\\
Distributed Algorithms/Systems\\
Computer Networking\\
Relational Databases\\
Information Retrieval\\
Information Security\\


\newpage

\section*{\textnormal{EXPERIENCE}}

\subsection*{美国国家超级计算机应用中心 (NCSA)}
\textit{科研软件工程师/科研助理, 可视化分析组, 视觉部门}\\
Aug. 2022 - Present \textbullet\ Urbana, IL, US
\begin{itemize}
\item 为多个国立卫生研究院 (NIH) 资助的科研项目开发应用, 主要负责生物医学及地理空间数据的可视化.
\item 使用 OpenLayers, GeoJSON 与人口普查局数据制作了伊利诺伊州蜱虫数据的交互式地图可视化.
\item 负责 INHS-MEL 实验室 BiteMap 蜱虫数据众包项目 web 应用的全栈开发与部署.
\item 参与了癌症演化树可视化分析工具 PhyloDiver 前端应用的维护与开发.
\end{itemize}

\subsection*{谷露软件}
\textit{前端工程师, 申请人跟踪系统(ATS)}\\
Nov. 2020 - Nov. 2021 \textbullet\ Shanghai, China
\begin{itemize}
\item 在多个项目团队内开发了多个web应用、微信小程序、前端库和后端渲染服务.
\item 参与了高度可定制的ATS系统底层的拖拽式图形用户界面生成器的维护与开发.
\item 负责公司首个单点登录系统(SSO)的前端界面和多个项目前端登陆流程的整合.
\end{itemize}

\subsection*{Étude LLC}
\textit{软件工程师, 创始团队}\\
Aug. 2019 - Aug. 2020 \textbullet\ Atlanta, GA, US
\begin{itemize}
\item 负责开发智能PDF阅读器的图书目次解析器 (准确率>85\%) 和滑词高亮等多个主要功能.
\end{itemize}

\section*{\textnormal{SKILLS}}

\subsection*{Languages}
C, C++, TypeScript, JavaScript, GLSL, HTML, SVG, CSS, Sass, LESS, PHP, Bash, Zsh, Python, Java, SQL, MATLAB, Haskell, OCaml, Common Lisp, Swift, Rust, WebAssembly

\subsection*{Frameworks / Libraries}
React, Angular, Node.js, Electron, Next.js, Redux, MobX, ECharts, NumPy, SciPy, PyTorch, OpenCV, SwiftUI, UIKit

\subsection*{Development Tools}
Visual Studio Code, Vim, Git, Make, LLVM, esbuild, Babel, Rollup, webpack, Parcel, Vite, Tailwind CSS, Cypress, Jest, Puppeteer, ESLint, Prettier, TypeDoc, GitHub Actions, GitLab CI, Docker, Kubernetes

\end{document}
